% Basear-se em https: //www.researchgate.net/profile/Carlos-Pantoja-3/publication/311743676_Aplicando_Sistemas_Multi-Agentes_Ubiquos_em_um_Modelo_de_Smart_Home_Usando_o_Framework_Jason/links/5859376008ae64cb3d493525/Aplicando-Sistemas-Multi-Agentes-Ubiquos-em-um-Modelo-de-Smart-Home-Usando-o-Framework-Jason.pdf

%Para a seção Trabalhos Relacionados. Nada mais é do que buscar soluções parecidas na web e referenciar aqui
\section{Trabalhos Relacionados} \label{sec:related}

A mensuração da oxigenação no sangue com dispositivos IOT não é novidade no meio acadêmico. No artigo \cite{trabalho1} é mostrado um monitoramento acessível que além de medir o SpO2 e frequência cardíaca, também mede a temperatura corporal. O sistema é baseado em um arduino e os sensores. Os resultados são obtidos em um LCD. Os parâmetros desse "oxímetro" são trasmitidos via Bluetooth para dispositivos Android e usando Wi-Fi como plataforma de internet.

Já em \cite{10783228} mostra que já existem soluções perfeitas no mercado, mas mesmo assim, os médicos detectaram alguns problemas que dificultam a operação desses dispositivos devido a fatores externos ou dos próprios pacientes. Esse artigo propõe novos sensores de baixo custo e preço de qualidade. A proposta é determinar se o dispositivo está sendo utilizado corretamente e enviar a um aplicativo para fazer uma fácil monitaração e determinar se o paciente escolhido sofre de algum problema.