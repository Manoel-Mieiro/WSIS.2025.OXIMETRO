% \section{Métodos}\label{sec:methods}


A presente pesquisa adota uma abordagem quantitativa, de natureza descritivo-experimental, por meio da qual se busca mensurar os níveis de saturação periférica de oxigênio (SpO$_2$) e verificar o funcionamento do protocolo de comunicação Multi-Agente desenvolvido em nível lógico. O estudo baseia-se na construção de um protótipo funcional de oxímetro, utilizando Arduino UNO, e no desenvolvimento de um sistema lógico de controle e monitoramento escrito na linguagem AgentSpeak, compilado e executado na ferramenta ChonIDE. A coleta de dados ocorre diretamente no protótipo físico, enquanto o processamento e a interpretação das informações são realizados pelo programa denominado \textit{\textbf{Oximeter}}. Essa configuração permite avaliar, de forma objetiva e controlada, a capacidade de comunicação entre hardware e software e a fidelidade das informações apresentadas no console, validando o comportamento esperado do sistema frente às condições experimentais simuladas.