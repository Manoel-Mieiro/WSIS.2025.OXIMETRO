\begin{abstract}
This work proposes the application of the \textit{Multi-Agent System} (\textit{MAS}) paradigm to enhance the autonomy of oximeters, enabling intelligent data collection, management, and notification of critical SpO$_2$ deficiency conditions to support networks, aiming for more efficient patient monitoring. The solution was developed using the \textit{Jason} framework in conjunction with the \textit{ARGO} architecture, which manages the hardware based on an \textit{ATMEGA} microcontroller (\textit{Arduino}). The study assessed the feasibility of this approach through a functional prototype built with a society of agents whose perceptions were derived from the Arduino microcontroller.
\end{abstract}

\begin{resumo}
Este trabalho propõe a aplicação do paradigma de Sistemas Multi-Agentes (SMA) para conferir maior autonomia a oxímetros, permitindo a coleta inteligente, o gerenciamento e a notificação de quadros críticos da deficiência de SpO$_2$ a redes de apoio, para acompanhamento mais eficiente do quadro. A solução foi desenvolvida com o uso do framework Jason, em conjunto com a arquitetura ARGO, responsável pelo controle do hardware baseado no microcontrolador ATMEGA (Arduino). O estudo aferiu a viabilidade desse tipo de aplicação por meio do protótipo construído, o qual fez uso de uma sociedade de agentes, cujas percepções advinham do microcontrolador Arduino.
\end{resumo}