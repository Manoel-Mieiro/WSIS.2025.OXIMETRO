\section{Introdução}\label{sec:introdução}
Fortis e Nora (2000) \cite{fortis2000hipoxemia} definem hipoxemia como a redução da oxigenação arterial (CaO$_2$), que é influenciada por meio das variáveis de saturação arterial de oxigênio (SpO$_2$), a concentração de hemoglobina (Hb) e a afinidade do oxigênio à Hb, sendo interpretado como a facilidade com que o oxigênio se liga ou se desprende da molécula de Hb. Collins et al. (2015) \cite{collins2015relating} afirmam que a hipoxemia contribui para a  redução na produção de energia nas células, o que compromete o funcionamento adequado de tecidos e órgãos. Em seu trabalho, destaca como métodos determinantes de hipoxemia, com destaque para os exames de sangue e uso de oxímetro de dedo; sendo o último o único não invasivo.

Oxímetros de dedo, além de não invasivos, apresentam o menor tempo de diagnóstico \cite{collins2015relating}; mas apostam na autonomia do operador. Então, tratando-se de um usuário leigo (paciente), não agrega valor ao processo de identificação da deficiência de saturação do oxigênio, já que o operador não é capaz de emitir parecer técnico sobre as informações exibidas no aparelho. Dessa forma, é necessário acompanhamento ativo por um profissional qualificado, uma vez que esses dados ficam retidos no local de medição. Diante disso, este artigo busca comprovar a viabilidade na aplicação de Sistemas Multi-Agente no monitoramento, gestão e propagação das medidas obtidas de pacientes hipoxêmicos em oxímetros, permitindo diagnóstico e acompanhamento técnico integral por meio de mecanismos de comunicação Multi-Agente. Para isso, foi adotada uma arquitetura em camadas, na qual cada módulo possui uma responsabilidade única — desde a coleta dos dados até a triagem e notificação — organizada como uma sociedade de agentes voltada ao acompanhamento do paciente hipoxêmico. A solução foi construída por meio da codificação do SMA, fazendo uso da \textbf{ChonIDE}, conectado-a a um microcontrolador Arduino UNO. Este realiza a leitura dos níveis de SpO$_2$ do paciente, com valores inseridos experimentalmente via um teclado LCD acoplado ao controlador.

O presente trabalho apresenta potencial contribuição para a área médica, uma vez que o monitoramento remoto contínuo fornecido pelo SMA-oxímetro pode reduzir a ocupação de unidades de emergência ao possibilitar o diagnóstico precoce de hipoxemia. Ademais, o sistema diminui a sobrecarga do médico responsável, cuja atuação presencial passa a ser exigida apenas em situações críticas identificadas automaticamente pelo sistema de triagem e mensageria.

O restante deste artigo está organizado da seguinte forma: a Seção 2 apresenta uma breve descrição de alguns conceitos que podem não ser de conhecimento do leitor. A Seção 3 descreve o estudo de caso desenvolvido neste trabalho. A Seção 4 conclui com as considerações finais e trabalhos futuros.