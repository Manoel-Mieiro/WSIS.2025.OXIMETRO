\section{Conclusão e Trabalhos Futuros}
O paradigma Multi-Agente foi aplicado de forma adequada ao domínio do problema, conforme demonstrado no protótipo funcional. A chave para esse resultado foi a simplicidade da implementação, cujo objetivo consistiu em integrar o microcontrolador Arduino a uma sociedade de agentes para automação do monitoramento de hipoxêmicos. Esse propósito foi alcançado por meio da criação de uma arquitetura em camadas de responsabilidade única, simulando a triagem hospitalar, com etapas de leitura, diagnóstico e acompanhamento. Embora simples, a validação da viabilidade de implantação abre caminho para trabalhos futuros com \textit{hardware} mais sofisticado, concebidos sob o mesmo paradigma, o que se mostra promissor ao se considerar a escalabilidade da solução para modelos de Inteligência Artificial, especialmente no contexto da IoT.

Como trabalhos futuros, pretende-se incluir sensores para detecção direta de SpO$_2$, dispensando a inserção manual do dado; desenvolver mecanismos de notificação mais robustos, como WhatsApp e ligações, para aprimorar a comunicação com a rede de apoio; e empregar meios de armazenamento dos dados coletados, possibilitando análises históricas do paciente, o que pode contribuir para a identificação das causas da hipoxemia. Além disso, serão conduzidas avaliações por meio de questionários de aceitação tecnológica, de forma a verificar a utilidade da proposta, sempre considerando os aspectos éticos envolvidos no processo.
