\section{Conclusão e Trabalhos Futuros}
Este artigo visou demonstrar a viabilidade do emprego de SMAs na medição dos níveis de saturação no sangue. A abordagem foi dividida em camadas de responsabilidade única: monitor, processador e mensageiro. Essas camadas representam uma sociedade de agentes, em conformidade com o paradigma Multi-Agente; guiadas em função de um objetivo: o acompanhamento integral de pacientes deficientes em SpO$_2$ no sangue. A partir do estudo, foi possível observar que SMAs são capazes de contribuir no acompanhamento de hipoxêmicos, empregando a funcionalidade de notificação quando necessário. No entanto, foi identificada a necessidade de modernizar o hardware utilizado neste trabalho, de forma a garantir maior precisão nas leituras e otimizar a comunicação entre os agentes.

Como trabalhos futuros, pretende-se incluir sensores de detecção de SpO$_2$, dispensando inserção manual desse dado - mecanismos de notificação mais robustos como WhatsApp e ligações, para melhor comunicação com a rede de apoio. Além disso, é pretendido empregar meios para armazenar os dados coletados, para eventuais análises de histórico do paciente, permitindo assim traçar melhor seu perfil e contribuir para a identificação das causas do quadro de hipoxemia. Por fim, é pretendido realizar avaliações por meio de questionários de aceitação de tecnologia para verificar a utilidade da proposta, sempre considerando todos os aspectos éicos que podem envolver tal processo.